\section{Results}

The Ziro package has been successfully published to the npm registry, with the latest version being \cc{0.0.16}\footnote{\url{https://www.npmjs.com/package/ziro}} as of now. The package is actively being developed and will likely see future releases with additional features.

\subsubsection*{Current Features and Capabilities}

The core functionality of Ziro is well-developed and includes several key features that have been tested and implemented:

\begin{itemize}
	\item \textbf{SSR, CSR, and Partially-SSR Modes}: The three rendering modes, Server-Side Rendering (SSR), Client-Side Rendering (CSR), and partially-SSR, are fully developed and tested. This allows developers to choose the rendering strategy that best fits their project needs.
	\item \textbf{Server-Client Integration}: The integration between the server and client has been successfully implemented, enabling smooth communication and data transfer between the two.
	\item \textbf{Type Safety}: Loaders, middlewares, and actions are now type-safe, ensuring that developers can leverage TypeScript for more reliable code in addition to better developer experience.
	\item \textbf{Plugin System}: A basic plugin system has been implemented, allowing users to add custom pages and layouts. This system is designed to be extendable, with more flexibility and additional features planned in the near future.
	\item \textbf{Example Project}: A simple example application is available in the examples directory of the repository, demonstrating basic usage and providing a starting point for developers. The example can be accessed at: \url{https://github.com/narixius/ziro-vite}
	\item \textbf{Testing}: The \cc{ziro/router} has been tested using Vitest, ensuring that the core router works as expected and that future changes can be made with confidence.

\end{itemize}

It is important to note that only the development environment has been implemented so far. The production build process for Ziro-based projects has not yet been completed. Future development will focus on extending the framework to support production builds, ensuring that projects built with Ziro can be deployed on various environments.



\pagebreak
