\section{Technical Specifications} % 10 pages


\subsection{Understanding the Router and Route Structure}

Ziro's router is a fundamental component that handles routing and the organization of routes and endpoints in an application. It plays a crucial role in navigating between different parts of an application, ensuring data flows seamlessly, and integrating various elements such as middleware, actions, and components. Understanding the router and route structure in Ziro is essential for developing and maintaining applications efficiently.

\subsubsection{Overview of the Router's Role and Structure}
The router in Ziro manages the definitions of all routes, making sure that each route is properly configured to handle requests and load data. The router tree is built on top of the \textbf{rou3}\footnote{\url{https://unjs.io/packages/radix3}} library, a lightweight and fast router for JavaScript based on a radix tree structure. This library provides an efficient way to manage and traverse routes.

\subsubsection{What is a Radix Tree?}
A radix tree (or prefix tree) is a type of data structure that is particularly well-suited for handling route paths. Each route is stored as a sequence of characters in the tree, allowing for efficient prefix matching and lookup. This structure is ideal for route trees as it reduces the number of comparisons needed when finding a specific route and supports hierarchical relationships.

\subsubsection{Main Properties of a Route}
\begin{itemize}
	\item \textbf{id}: The path of the route also acts as the identifier because it is unique.
	\item \textbf{parent}: The parent route.
	\item \textbf{loader}: A function or method responsible for loading data needed by the route.
	\item \textbf{middlewares}: Functions that run before the route loader and actions to perform tasks such as authentication, data validation, or request modification.
	\item \textbf{actions}: Functions that handle post requests, often used in handling form requests.
	\item \textbf{meta}: Metadata associated with the route, used for SEO.
	\item \textbf{props}: Properties that relates to the route. Often used for storing the rendering-related data, including the main \texttt{Component}, \texttt{ErrorBoundary}, and \texttt{Loading} components (in React concept).
\end{itemize}

\subsubsection{Parent-Child Route Relationships}
Routes in Ziro can have parent-child relationships that help structure the application in a nested way. For example, a layout route has some sub-routes that inherit the data from the parent, allowing for shared layouts, middleware and loading data. This nesting supports complex page structures and maintains consistent data flow across different parts of the application.

\pagebreak

\subsubsection{Example of a Basic Router and Route Definition}
Consider the following example of how a router with different routes are defined in Ziro:

\begin{figure}[h!]
\begin{minted}{javascript}
import { Route, Router } from 'ziro/router';

const router = new Router()

const rootRoute = new Route("_root", {
  loader: async () => {
    return fetch("https://api.example.com/data").then(res => res.json())
  }
});

const indexRoute = new Route("/", {
  parent: rootRoute,
  loader: async () => {
    return fetch("https://api.example.com/index").then(res => res.json())
  }
});
router.addRoute(indexRoute)

export default router;
\end{minted}
\caption{Example of a Basic Router and Route Definition}
\end{figure}

In this example, the router is created (line 3), and two routes (\cc{rootRoute} and \cc{indexRoute}) are defined. Only the \cc{indexRoute} is added to the router as it is the main route visible to users. The routes ending with \cc{\_root} and \cc{\_layout} are considered non-viewable routes and serve as wrappers around other routes. When the router is called to find the route tree for a specific request, such as the "/" endpoint, it should return an array with the \cc{rootRoute} as the first item and the \cc{indexRoute} as the second item. This calculation occurs via the \cc{addRoute} method, which builds the route tree recursively, traversing the parent property until it reaches undefined and adds the route array as the value of the radix tree for the given endpoint.

\pagebreak

\subsubsection*{Minimal Example of How the Router Creates the Radix Tree}
\begin{figure}[h!]
\begin{minted}{javascript}
import * as rou3 from 'rou3'

class Router {
  radixTree;
  constructor(){
    this.radixTree = rou3.createRouter();
  }

  addRoute(route){
    let tree = [route]
    let routePath = route.id;

    while (true) {
      if (!tree[0].getParent()) break
      const parentRoute = tree[0].getParent()
      if (parentRoute) tree.unshift(parentRoute)
    }

    rou3.addRoute(this.radixTree, '', routePath, tree)
  }
}
\end{minted}
\caption{Minimal Example of How the Router Creates the Radix Tree}
\end{figure}

This code snippet shows how the router class initializes a \textbf{rou3} radix tree and adds routes by traversing their parent routes recursively to build a comprehensive route tree structure.





\subsubsection{How the Router Handles Requests}

When the \cc{router.handleRequest(request)} method is called, the router follows a \hl{systematic} process to find and handle the appropriate routes. Here is how the process works:

\begin{enumerate}
  \item \textbf{Request Matching}: The handleRequest() method uses the radix tree to match the incoming request path to a specific sequence of routes. This is done by calling \cc{rou3.findRoute(this.radixTree, '', String(request.url.pathname))}

  \item \textbf{Load Routes}: Once a route tree is matched, the routes should be start handling the incoming request. The middlewares and the route handlers (loaders or actions) of each route will be called in order from the parent to the child sequentially. The middlewares and route handles (loaders or actions) can intercept the request, and prevent it from proceeding by throwing an early response.

  \item \textbf{Store the loaded data in the cache}: After calling each loader or middleware the returned data will be cached in a Cache object to be used in the rendering process and prevent data loss.

\end{enumerate}

After router find and loads a route tree, it's ready to be rendered on the page using a rendering library. In this project we have the \cc{ziro/react} to render the router in the React environment.


\subsubsection{Web standards}
% \begin{hintbox}[label={hint:example}]{Web standards}
% \end{hintbox}

The Ziro Router is built on top of web standards, specifically the \cc{Request} and \cc{Response} objects. These objects are part of the Fetch API, which is a standard way to handle HTTP requests and responses in JavaScript. By leveraging these web standards, Ziro ensures compatibility and consistency across different environments, whether it's running on the client-side in a browser or server-side in a Node.js environment. Every loader, action or middleware described bellow has access to the request/response objects (based on their responsibility) to handle the incoming request and the response object to intercept the generated response and do proper action if needed.

% \subsubsection*{Benefits of Using Web Standards}
% \begin{itemize}
%   \item \textbf{Consistency}: Using \cc{Request} and \cc{Response} objects ensures that the same code can handle HTTP requests and responses in a consistent \hl{manner} across different environments.
%   \item \textbf{Interoperability}: Web standards are widely supported, making it easier to integrate with other libraries and frameworks that also \hl{adhere} to these standards.
%   \item \textbf{Future-Proofing}: As web standards evolve, Ziro can take advantage of new features and improvements without needing significant changes to its core architecture.
% \end{itemize}


\subsubsection{Loaders}
Loaders are essential functions for building dynamic and data-driven applications with Ziro. They help manage data fetching and preparation before a route’s main content is rendered. Understanding how to use loaders is key to building applications that are both efficient and maintainable.

\subsubsection*{What are Loaders in Ziro?}
Loaders in Ziro are functions \hl{associated} with routes that handle data fetching and preparation. They are especially valuable for server-side data loading, enabling data to be fetched from external APIs, databases, or other sources. Loaders are executed after the middlewares registered on the route and before starting the process of loading the nested routes.

\pagebreak
\subsubsection*{Example of a Simple Loader}
Here is a simplified example to illustrate how a loader function works in Ziro:
\begin{figure}[h!]
\begin{minted}{javascript}
import { Router, Route } from 'ziro/router';

const router = new Router()

const usersRoute = new Route('/users/:userId', {
  loader: async ({ params }) => {
    // Fetch user data from an API
    const response = await fetch(`https://api.com/users/${params.userId}`);
    if (!response.ok) {
      throw new Error('Failed to load user data');
    }
    const userData = await response.json();

    // Return the user data to store in the cache
    return {
      name: userData.name,
      email: userData.email,
      avatar: userData.avatar,
    };
  };
});

router.addRoute(router);
\end{minted}
\caption{Example of a Simple Loader}
\end{figure}


\subsubsection{Middlewares}
Ziro middleware is a powerful tool used to handle various tasks before or after a route's request and response cycle. Middleware functions run in sequence and can be used to modify requests, handle authentication, log information, or perform other pre- and post-processing actions. Here, we will discuss the purpose of middleware and present a simple example to illustrate how it works.

\subsubsection*{What is Middleware in Ziro?}
Middleware in Ziro is a function that can intercept and modify requests and responses at different points in the lifecycle of a request. It allows for reusable logic that can be shared across multiple routes or endpoints. Middleware can be executed at various stages, such as before a request is processed (\cc{onRequest}) or before a response is sent (\cc{onBeforeResponse}).

\subsubsection*{Simple Middleware Example}
To understand how middleware works in Ziro, let’s look at a simplified example:
\begin{figure}[h!]
\begin{minted}{javascript}
import { Middleware } from 'ziro/router';

// Simple middleware that logs the request URL and method
export const simpleLogger = new Middleware('simple-logger', {
  async onRequest({ request }) {
    console.log(`Request received: ${request.method} ${request.url}`);
  },
  async onBeforeResponse({ response }) {
    console.log(`Response status: ${response.status}`);
  },
});
\end{minted}
\caption{Simple Middleware Example}
\end{figure}

\subsubsection*{Explanation:}

\begin{itemize}
  \item \textbf{Import Statement}: We import Middleware from the \cc{ziro/router} module.
  \item \textbf{Middleware Creation}: We create a new instance of Middleware, passing in a name ('simple-logger') and an object containing the \cc{onRequest} and \cc{onBeforeResponse} functions.
  \item \textbf{\cc{onRequest} Function}: This function runs before the request is processed. In this case, it logs the HTTP method and the request URL to the console.
  \item \textbf{\cc{onBeforeResponse} Function}: This function runs before the response is sent. It logs the status code of the response.
\end{itemize}

\subsubsection*{How Middleware Fits into the Router Lifecycle?}
Middleware functions in Ziro can be attached to routes and are executed in the order they are added. When a request is made, the following sequence occurs:
\begin{enumerate}
  \item The \cc{onRequest} method of each middleware is executed in the order they are defined.
  \item The route’s handler (the loader or action) is executed to process the request and prepare a response. During this step, the current route's handler (the loader or action) runs, followed by any handler from nested routes. Middleware functions that are defined for these routes are also executed, running in sequence as per the route's configuration. This ensures that all necessary data is fetched and any pre-processing needed for the request is completed before generating the final response.
  \item The \cc{onBeforeResponse} method of each middleware runs in the reverse order before the response is sent to the client.
\end{enumerate}
This enables developers to control and modify both the incoming request and outgoing response with ease.

\pagebreak
\subsubsection*{More Complex Middleware Example}
For more advanced use cases, middleware can be used to implement features such as logging, authentication, or performance tracking. Here is a more detailed example:
\begin{figure}[h!]
\begin{minted}{javascript}
import { Middleware } from 'ziro/router';

// Middleware that logs request details and response time
export const requestLogger = new Middleware('request-logger', {
  async onRequest({ dataContext }) {
    dataContext.responseTime = Date.now();
  },
  async onBeforeResponse({ request, dataContext }) {
    const responseTime = Date.now() - dataContext.responseTime;
    const pathname = new URL(request.url).pathname;
    console.log(`Request to ${pathname} took ${responseTime}ms`);
  },
});
\end{minted}
\caption{More Complex Middleware Example}
\end{figure}

In this example, the response time of the request will be logged by storing the current time on the \cc{dataContext} (in the \cc{onRequest} method) and will be logged right after the child routes loading completed (the \cc{onBeforeResponse} method). The \cc{dataContext} object is used to be a shared data layer through the loading the routes sequentially.

\pagebreak
\subsubsection*{Example of Integrating Loader and Middlewares in a route}
Here is an integrated example of how loaders and middlewares can be used in a route definition:

\begin{figure}[h!]
\begin{minted}{javascript}
import { Router, Route } from 'ziro/router';
import { requestLogger } from './middlewares/request-logger'

const router = new Router()

const usersRoute = new Route('/users/:userId', {
  middlewares: [requestLogger],
  loader: async ({ params }) => {
    // Fetch user data from an API
    const response = await fetch(`https://api.com/users/${params.userId}`);
    if (!response.ok) {
      throw new Error('Failed to load user data');
    }
    const userData = await response.json();

    // Return the user data to store in the cache
    return {
      name: userData.name,
      email: userData.email,
      avatar: userData.avatar,
    };
  };
});

router.addRoute(router);
\end{minted}
\caption{Example of Integrating Loader and Middlewares in a route}
\end{figure}

This example demonstrates how to use middleware on a route that has a loader. The middleware, named \cc{requestLogger} (Figure 5), logs the request details and calculates the time taken by the loader to fetch data from the API. This is achieved by defining an \cc{onRequest} function that stores the start time and an \cc{onBeforeResponse} function that logs the request URL, method, and the total response time.


\subsubsection{Actions}
Ziro actions are an essential part of building dynamic applications, allowing developers to handle complex logic that goes beyond simple data fetching or rendering. Actions in Ziro provide a mechanism to handle server-side operations, such as form submissions, data updates, and other tasks that require request handling and processing. By using actions, developers can create robust, interactive applications that manage data more efficiently and ensure that certain operations are secure and well-structured.

\subsubsection*{What are Actions in Ziro?}
Actions in Ziro are special functions that can be associated with specific routes and used to handle mutation requests like POST HTTP requests. These functions are designed to handle tasks such as modifying data, performing validations, or managing interactions that involve user input. An action can be defined with an input validation schema and a handler function to process the request body.


\subsubsection*{Benefits of Using Actions}
\begin{itemize}
  \item \textbf{Input Validation}: Actions use schemas to validate the input data before processing. This ensures that only valid data is handled, preventing errors and enhancing security.

  \item \textbf{Centralized Logic}: Actions help in maintaining clear and modular code by placing related logic in one place, making the codebase easier to understand and maintain.

  \item \textbf{Customizable Behavior}: Each action can be customized to handle different types of requests and data, providing flexibility in how data is managed and manipulated.

\end{itemize}

\subsubsection*{What is a Schema in Actions?}
A schema in actions is a data validation structure that defines the shape and constraints of the input data for actions. Schemas are crucial for ensuring that only valid data is processed, which helps prevent errors and enhances the security and reliability of the application. In Actions, schemas should be defined using \textbf{Zod}\footnote{\url{https://zod.dev}}, a TypeScript-first schema declaration and validation library.

\subsubsection*{What is Zod?}
\textbf{Zod} is a TypeScript-first schema validation library that allows developers to define and validate data structures in a type-safe manner. With Zod, developers can create complex validation logic that is easy to read and maintain. It supports various data types and provides built-in methods for validating strings, numbers, arrays, objects, and more. The integration of Zod in actions makes it easier to enforce data integrity and ensure that actions receive correctly formatted input.

\subsubsection*{Benefits of Using Schemas in Actions}

\begin{itemize}
  \item \textbf{Type Safety}: Schemas are defined in TypeScript, providing strong type checking at compile time, which helps catch errors early in the development process.

  \item \textbf{Input Validation}: Using Zod schemas, actions can validate incoming data to make sure it meets the required conditions before processing.

  \item \textbf{Error Handling}: Schemas can specify detailed error messages, making it easier to understand why an input is invalid and guiding developers to fix issues efficiently.

  \item \textbf{Consistency}: Schemas ensure that data structures are consistent across different parts of the application, enhancing code maintainability.
\end{itemize}

You can find an example of actions in the figure 7

\begin{figure}[h!]
\begin{minted}{javascript}
import { Route, Action } from 'ziro/router';
import { z } from 'zod';

let todos = [];

export const actions = {
  addTodo: new Action({
    input: z.object({
      title: z.string().min(3, 'Title must be at least 3 characters'),
    }),
    async handler(body, ctx) {
      todos.push({
        ...body,
        isDone: false,
      });
      return {
        ok: true,
      };
    },
  }),
  toggleTodo: new Action({
    input: z.object({
      index: z.coerce.number().min(0, 'This field is required'),
    }),
    async handler(body) {
      todos[body.index].isDone = !todos[body.index].isDone;
      return {
        ok: true,
      };
    },
  }),
  deleteTodo: new Action({
    input: z.object({
      index: z.coerce.number().min(0, 'This field is required'),
    }),
    async handler(body) {
      todos.splice(body.index, 1);
      return { ok: true };
    },
  }),
};

export const todoRoute = new Route("/todo", {
  loader: async () => todos,
  actions: actions
})
\end{minted}
\caption{Example of Actions in a To-Do List App}
\end{figure}
\pagebreak

\subsubsection{Meta Functions}
Meta functions are an important part of creating well-rounded web applications with Ziro. They help manage the meta-information of routes, such as page titles, descriptions, and other SEO-related data. By defining meta functions for routes, developers can enhance the discoverability and ranking of their web pages, improve social media sharing, and provide a better overall user experience.

\subsubsection*{What are Meta Functions in Ziro?}
Meta functions in Ziro are functions that are associated with specific routes and are responsible for setting metadata in the HTML head. These functions receive data from the route’s loader, allowing them to dynamically generate metadata based on the content of the page. Meta functions can be defined to return various pieces of meta-information, such as the page title, description, and other elements needed for SEO and social sharing.

Ziro’s integration with the \textbf{unhead}\footnote{\url{https://unhead.unjs.io/}} library makes it possible to manage meta tags in a reactive and dynamic manner, ensuring that the content in the HTML head is always up-to-date with the current route.


\subsubsection*{Benefits of Using Meta Functions}
\begin{itemize}
  \item \textbf{Improved SEO}: Properly configured meta functions ensure that search engines can better understand the content of each page, improving search rankings.

  \item \textbf{Enhanced Social Sharing}: Meta tags help generate rich previews on social media platforms when links are shared.

  \item \textbf{Dynamic and Context-Aware}: Meta functions can use data loaded by route loaders to generate context-specific metadata.

  \item \textbf{Consistency}: Using meta functions ensures that each route has consistent and up-to-date metadata.
\end{itemize}

\pagebreak
\subsubsection*{Example of a Meta Function}
\begin{figure}[h!]
\begin{minted}{javascript}
import { Route, Action } from 'ziro/router';
import z from 'zod'

const todos = []

const meta = async ({ loaderData }) => {
  return {
    title: `${loaderData.todos.length} items in list | To-Do App`,
    description: `Track your tasks for free!`,
  };
};

const loader = async () => {
  return { todos };
}

const actions = {
    addTodo: new Action({
      input: z.object({
        title: z.string().min(1, 'Title is required')
      }),
      async handler(body){
        todos.push(body);
      }
    })
}

const todoRoute = new Route("/todo", {
  meta,
  loader,
  actions
})
\end{minted}
\caption{Example of a Meta Function}
\end{figure}

\pagebreak

\subsubsection{Transitioning to Framework Development}

While Ziro's core libraries provide powerful capabilities for handling routing, data loading, and meta management, building a complete framework requires integrating these functionalities within a \hl{cohesive} development environment. This is where a modern bundler like \textbf{Vite} comes into play. Vite allows Ziro to efficiently handle dynamic module loading, and streamline development workflows. One of the key advantages of using Vite is the ability to implement a file-based routing system, \hl{eliminating} the need to define routes manually. This approach simplifies the development process by allowing routes to be automatically generated based on the directory structure of route files. By \hl{leveraging} Vite, Ziro evolves from a routing libary  into a comprehensive framework capable of delivering web applications.


\subsection{Vite Overview}

Vite is a next-generation frontend tooling system that provides a fast and efficient development environment for modern web applications. By leveraging features like on-demand module loading, hot module replacement (HMR), and support for server-side rendering (SSR), Vite creates a seamless experience for developers building frameworks like Ziro.

\subsubsection*{Why Vite Enables Framework Development}
\begin{itemize}
	\item \textbf{Dynamic Module Loading}: Vite’s \cc{ssrLoadModule} method is a game-changer for frameworks. It allows modules to be dynamically loaded during SSR, making it possible to execute Javascript on-demand and generate updated values on the server based on latest file changes.
	\item \textbf{Hot Module Replacement (HMR)}: HMR drastically reduces development time by enabling instant updates to modules without requiring a full page reload. This feature is invaluable when working on a project, where frequent changes to files need immediate feedback.
	\item \textbf{File Watching and Auto Reload}: Vite’s built-in file-watching mechanism ensures that any changes to the project’s codebase are automatically detected. Combined with its optimized reload system, Vite provides a smooth development experience even for large-scale applications.
	\item \textbf{Enhanced SSR Capabilities}: Vite simplifies the implementation of server-side rendering by providing tools to manage server and client-side modules separately. This separation ensures that Ziro’s SSR functionality remains maintainable.
	\item \textbf{Plugin System}: Vite’s extensible plugin system allows Ziro to integrate additional functionality, such as generating route-related files (\cc{ziro/generator}) and handling file-based routing.
	\item \textbf{Vite is Fast}: By utilizing native ES modules (ESM), Vite removes the need for traditional bundling during development. This results in lightning-fast startup times, even for complex projects.
\end{itemize}

In summary, Vite’s robust feature set and modern approach to tooling make it an ideal choice for bootstrapping a development server for Ziro. It not only accelerates the development process but also ensures that the resulting framework is scalable and maintainable.

\subsection{Vite Plugins and Their Role in Ziro}

Vite plugins are essential tools that extend the functionality of Vite, allowing developers to customize and automate parts of the development and build process. In the context of Ziro, Vite plugins play an important role in detecting file changes, generating route-related files and generating server-side rendered pages.

\subsubsection{How Vite Plugins Work}
A Vite plugin is essentially a JavaScript function or object that hooks into various stages of the Vite lifecycle\footnote{\url{https://vite.dev/guide/api-plugin}}, such as server startup, file changes, and builds. These plugins use Vite's robust API to perform tasks like file processing, module transformation, and custom behavior during development or production builds.

\subsubsection{File Watching and Route Detection}
Vite comes with a file-watching system based on the chokidar\footnote{\url{https://github.com/paulmillr/chokidar}} library, which monitors the file system for changes. This feature allows Ziro’s Vite plugin to:
\begin{itemize}
	\item \textbf{Track Route Files}: By watching specific directories (like \cc{pages/}), the plugin can detect when a new route file is added, modified, or removed.
	\item \textbf{Trigger Updates}: When changes are detected, the plugin triggers the generation of files necessary for the router to work correctly.
\end{itemize}

\subsubsection{Generating Route-Related Files}
When a change is detected in the route files, Ziro’s Vite plugin takes the following steps to ensure the router remains up to date:

\begin{enumerate}
	\item \textbf{Manifest Generation}: A \cc{manifest.json} file is created, containing information about each route, such as its path, associated middleware, loaders, actions, etc...
	\item \textbf{Server and Client Router Files}: Separate router files are generated for server-side (\cc{router.server.ts}) and client-side(\cc{router.client.ts}) usage. The server router handles SSR-specific logic, while the client router is optimized for browser execution, including lazy loading\footnote{\url{https://react.dev/reference/react/lazy}} for route components.
	\item \textbf{TypeScript Definitions (\cc{routes.d.ts})}: The plugin generates a TypeScript file that defines types for all routes. This ensures type safety, making it easier for developers to work with route data in a consistent way.
\end{enumerate}

\subsubsection{Benefits of Vite Plugins in Ziro}
\begin{itemize}
	\item \textbf{Automation}: Developers don’t need to manually update route configurations or related files when they add or modify routes. The plugin handles everything automatically. It generates the route objects (\cc{new Router(...)}) based on the exports of the route files.
	\item \textbf{Efficiency}: By using Vite's built-in file watching and caching mechanisms, updates are fast and incremental, meaning only the affected files are regenerated.
	\item \textbf{Consistency}: The plugin ensures that all generated files, such as \cc{routes.d.ts}, are always in sync with the file system. This reduces the risk of bugs caused by mismatched configurations.
	\item \textbf{Dynamic File Transformation}: During the build or development process, the plugin can modify or transform route modules to exclude server-only logic or optimize for the client environment.
\end{itemize}

\subsection{Manifest Generation}

To maintain a single source of truth and support server/client-specific routers, Ziro generates a manifest file that encapsulates all route-related information. This manifest serves as a centralized repository, containing details about each route and its associated exports. It simplifies route management, type generations, and ensures that both server-side and client-side routers built on single source of truth.

\subsubsection{Why a Manifest is Necessary}

In file-based routing systems, routes are defined dynamically based on the structure of route files and their exported variables. The manifest acts as the backbone of this system by:

\begin{itemize}
	\item Storing metadata about each route file, including its path and exported properties.
	\item Facilitating the generation of server and client router files with consistent data.
\end{itemize}

\subsubsection{Structure of the Manifest}
A manifest is represented as a \textbf{JSON} object. Below is an example of a manifest for a simple application with a root, layout and a single page at \cc{/}:

\pagebreak
\begin{figure}[H]
\begin{minted}{json}
{
  "/_root": {
    "id": "/_root",
    "routeInfo": {
      "filepath": "pages/_root.tsx",
      "index": false,
      "hasActions": false,
      "hasComponent": true,
      "hasErrorBoundary": true,
      "hasLoader": true,
      "hasLoadingComponent": true,
      "hasMeta": true,
      "hasMiddleware": true,
      "hasLayout": true
    }
  },
  "/_layout": {
    "id": "/_layout",
    "routeInfo": {
      "filepath": "pages/_layout.tsx",
      "index": false,
      "hasActions": false,
      "hasComponent": true,
      "hasErrorBoundary": true,
      "hasLoader": true,
      "hasLoadingComponent": true,
      "hasMeta": true,
      "hasMiddleware": false,
      "hasLayout": true
    },
    "parentId": "/_root"
  },
  "/": {
    "id": "/",
    "routeInfo": {
      "filepath": "pages/index.tsx",
      "index": true,
      "hasActions": false,
      "hasComponent": true,
      "hasErrorBoundary": false,
      "hasLoader": true,
      "hasLoadingComponent": true,
      "hasMeta": true,
      "hasMiddleware": false,
      "hasLayout": false
    },
    "parentId": "/_layout"
  }
}
\end{minted}
\caption{Example of manifest with 1 route at \cc{/}}
\end{figure}

\pagebreak

\subsubsection{Components of the Manifest}
\begin{enumerate}
  \item \textbf{Route ID (\cc{id})}: A unique identifier for the route, corresponding to its path.
  \item \textbf{Route Information (\cc{routeInfo})}: Details about the route file, such as:
  \begin{itemize}
    \item \cc{filepath}: The relative path to the file in the project structure.
    \item \cc{index}: Indicates if the route is an index route.
    \item \cc{hasComponent}: Indicates if the file has a default export.
    \item \cc{hasLayout}: Indicates if the file has a export called 'layout'.
    \item \cc{hasErrorBoundary}: Indicates if the file exports a variable called \cc{ErrorBoundary} (a React specific rule).
    \item \cc{hasLoader}: Indicates if the file exports a variable called \cc{Loading} (a React specific rule).
    \item \cc{hasLoader},\cc{hasActions} \cc{hasMeta}, \cc{hasMiddleware}: Flags for other exported properties like loaders, actions, meta functions, and middleware.
  \end{itemize}

  \item  \textbf{Parent ID (\cc{parentId})}: Specifies the parent route for nested routes, enabling hierarchical routing.
\end{enumerate}

\subsubsection{How the Manifest is Generated}
\begin{enumerate}
  \item \textbf{File System Analysis}: During the development server's startup, Ziro uses Vite's file system watcher and the \cc{es-module-lexer}\footnote{\url{https://github.com/guybedford/es-module-lexer}} library to scan the project directory for route files.
  \item \textbf{Export Inspection}: For each route file, \cc{es-module-lexer} identifies the exported variables (e.g., \cc{loader}, \cc{meta}, \cc{actions}, \cc{middleware}, \cc{Loading}, \cc{ErrorBoundary}, and the \cc{default} component).
  \item \textbf{Manifest Construction}: The gathered data is structured into a JSON object following the manifest format.
\end{enumerate}

\subsubsection{Output of Manifest Generation}
Once the manifest is created, it serves as the input for generating \cc{router.server.ts} and \cc{router.client.ts}. These files contain optimized routing logic tailored for the server and client environments, respectively.

For example:
\begin{itemize}
  \item \cc{router.server.ts} runs only on the server and handles server-side routing, including loaders, actions and middleware.
  \item \cc{router.client.ts} runs on the browser and focuses on client-side navigation and lazy loading components.
\end{itemize}
The manifest ensures that both files are generated consistently, reflecting the same route structure.


\subsubsection{Why This Approach Works}
By using a manifest file, Ziro achieves:
\begin{itemize}
  \item Both server and client routers share a single source of truth.
  \item The manifest is generated automatically during development, reducing manual configuration.
  \item New routes and modifications are detected dynamically, updating the manifest and routers in real time.
\end{itemize}

This process forms the foundation of Ziro’s file-based routing system, enabling seamless integration of server and client functionality.

\textbf{Hint}: By extending the \cc{manifest.json} we can add custom pages or layouts out of the project directory to our app. It happens because the server/client router will be generated based on this manifest file. This is actually the basis of the Ziro Plugin which will be discussed further in the next sections.

\subsubsection{Example of a route serves at \cc{/}}
\begin{figure}[H]
\begin{minted}{javascript}
// pages/index.tsx

import { MetaFn } from 'ziro/router'

export const loader = async () => {
  return fetch('https://api.example.com/data').then(r => r.json())
}

export const meta: MetaFn<'/'> = async () => {
  return {
    title: 'Homepage',
  }
}

export default function Index() {
  return (
    <div>
      <h1>Home page</h1>
    </div>
  )
}

export const Loading = () => {
  return <span>Loading...</span>
}

export const ErrorBoundary = () => {
  return <span>Something went wrong, please reload the page.</span>
}
\end{minted}
\end{figure}
In this example, when this route is going to to be rendered on the browser, it shows the \cc{Loading} component while the data is being fetched and the \cc{ErrorBoundary} component if an error occurs during the loading process. The \cc{meta} function is used to set the page title to "Homepage" when the route is rendered on the browser. The \cc{loader} function fetches data from an API endpoint and returns it to be used in the component. The \cc{default} export is the main component that will be rendered when the route is loaded successfully.
\pagebreak

\subsection{Router Generation}

As described in the previous section, to ensure a seamless routing experience tailored for both client and server environments, Ziro generates two distinct router files based on the manifest: \cc{router.client.ts} and \cc{router.server.ts}. These files contain the necessary routing logic optimized for their respective environments.

\subsubsection{Route Naming Conventions}
The naming conventions for the route files in the pages directory determine how routes are generated and how layouts and contexts are applied. This structured convention is important on generating the routers. Below are the key conventions:

\textbf{Special Files}
\begin{itemize}
	\item \cc{\_root.tsx}: This file represents the root of the application. It is used to define the top-level contexts and layouts that will wrap the entire app. All routes in the application will inherit from the contexts or layouts defined here.
	\item \cc{\_layout.tsx}: This file acts as a layout wrapper for every route within the specific directory and its child directories. It is useful for defining shared components like headers, footers, or navigation menus that need to appear across multiple routes in the directory.
	\item \cc{index.tex}: This file represents the index route of the directory. It serves as the default route for the directory when no specific route is provided.
	\item \textbf{Dynamic routes}: Any file that begins with a colon (\cc{:}) generates a dynamic named route. For example, \cc{:slug.tsx} defines a route that matches URLs like \cc{/blog/post-slug}, where \cc{post-slug} can be any dynamic value. Dynamic routes are particularly useful for handling paths based on unique identifiers or variables in the URL.
\end{itemize}

\subsubsection{Example File Structure and Generated Routes}
Consider the following directory structure:

\begin{minted}{bash}
pages/
- _layout.tsx
- index.tsx
- blog/
  - _layout.tsx
  - index.tsx
  - :slug.tsx
\end{minted}

This structure generates the following index routes:
\begin{enumerate}
	\item \cc{/}: Wrapped by the root layout defined in \cc{pages/\_layout.tsx}.
	\item \cc{/blog}: Wrapped by both \cc{pages/\_layout.tsx} and \cc{pages/blog/\_layout.tsx}.
	\item \cc{/blog/:slug} A dynamic route where \cc{slug} is replaced with a specific value, such as \cc{post-1} or \cc{my-first-post}. Also wrapped by \cc{pages/\_layout.tsx} and \cc{pages/blog/\_layout.tsx}.
\end{enumerate}

The \cc{\_layout.tsx} files allow for nested layouts, where a directory-level layout wraps its routes and those in child directories. For instance, \cc{pages/\_layout.tsx} serves as the outermost layout for the app (after \cc{\_root.tsx} file, if defined), while \cc{pages/blog/\_layout.tsx} defines a layout for routes within the \cc{blog} directory, inheriting the root layout. This modular and scalable approach makes it easy to handle shared components, manage complex route hierarchies, and implement reusable layouts, ensuring a clean and predictable structure for building applications. Dynamic routes further enhance flexibility by allowing URL patterns to adapt to varying data.



\subsubsection{Client-Side Router (\cc{router.client.ts})}
The client-side router is designed to handle navigation and dynamic component loading in the browser. It leverages React’s \cc{lazy} for code splitting and ensures that components, meta functions, and loading/error boundaries are loaded only when needed.

\begin{figure}[H]
\begin{minted}{javascript}
import { Router, Route } from 'ziro/router';
import { lazy } from 'react';

const router = new Router({ "mode": "csr" });

const dynImport = () => import("../pages/index")

const pages_indexRoute = new Route("/", {
  meta: (...args) => dynImport().then(m => m.meta(...args)),
  loader: (...args) => dynImport().then(m => m.loader(...args)),
  props: {
    component: lazy(dynImport),
    ErrorBoundary: lazy(() => dynImport().then(m => ({ default: m.ErrorBoundary }))),
    LoadingComponent: lazy(() => dynImport().then(m => ({ default: m.Loading }))),
    Layout: lazy(() => dynImport().then(m => ({ default: m.Layout }))),
  },
});
router.addroute(pages_indexRoute);

// Similar logic for other routes...

export router;
\end{minted}
\caption{Example of router.client.ts}
\end{figure}

\textbf{Details}:
\begin{itemize}
  \item Components and route parts (\cc{meta}, \cc{loader}) are lazily loaded using \cc{import()} to optimize performance so only the required parts of the app are downloaded as users navigate between routes.
  \item Props defines the Component, Loading, and ErrorBoundary for each route.
\end{itemize}

\subsubsection{Server-Side Router (\cc{router.server.ts})}

The server-side router focuses on server-specific logic, such as loaders and middleware. This file is responsible for handling requests and preparing the data needed for server-side rendering.

\begin{figure}[H]
\begin{minted}{javascript}
import { Router, Route } from 'ziro/router';
import * as pagesIndex from '../pages/index';

const router = new Router({ "mode": "csr" });

const pagesIndexRoute = new Route("/", {
  meta: pagesIndex.meta,
  loader: pagesIndex.loader,
  props: {
    component: pagesIndex.default,
    ErrorBoundary: pagesIndex.ErrorBoundary,
    LoadingComponent: pagesIndex.Loading,
    Layout: pagesIndex.Layout,
  },
});
router.addroute(pages_indexRoute);

// Similar logic for other routes...

export router;
\end{minted}
\caption{Example of router.server.ts}
\end{figure}

\textbf{Details}:
\begin{itemize}
  \item All components and metadata are directly imported, as the server does not benefit from lazy loading.
  \item When router mode has set to \cc{ssr} or \cc{partially-ssr}, the server router file includes server-only logic like loaders for data fetching and middleware for request handling. In this case, the \cc{router.client.ts} doesn't include the server methods (loaders, middlewares or actions).
\end{itemize}

\subsubsection{Why Two Separate Router Files?}
The client router needs to be optimized for browser environment by importing the route modules dynamically, while the server router focuses on loading all required resources to handle SSR in a single file. To achieve this, we need to have 2 seperated files with totally different syntaxes. Also server and client have different requirements for resource loading and routing logic. Separating these concerns ensures a safer architecture as the server-logic code doesn't get leaked to the browser. For example, when the router mode is set to \cc{ssr}, the client router file doesn't include server-only logic like \cc{loaders} and \cc{middlewares}, which should be run on the server.

\subsection{TypeScript Integration}
During file generation chain of Ziro, a typescript declaration file will be generated called \cc{routes.d.ts}. It acts as the type declaration file for all routes in the application, ensuring that developers can rely on accurate type information for each route.

Here’s a sample of a generated routes.d.ts file:

\begin{figure}[H]
\begin{minted}{typescript}
import {
  Route,
  LoaderReturnType,
  IntersectionOfMiddlewaresResult,
  GetRouteDataContext
  } from 'ziro/router';
import * as pages_root from '../pages/_root';
import * as pagesIndex from '../pages/index';

declare module 'ziro/router' {
  interface RouteFilesByRouteId {
    "/_root": {
      route: Route<
        "/_root",
        LoaderReturnType<typeof pages_root.loader>,
        {},
        typeof pages_root.middlewares,
        undefined
      >;
      dataContext: {} & IntersectionOfMiddlewaresResult<
        typeof pages_root.middlewares
      >;
    };
    "/": {
      route: Route<
        "/",
        LoaderReturnType<typeof pagesIndex.loader>,
        {},
        [],
        RouteFilesByRouteId["/_root"]["route"]
       >;
      dataContext: GetRouteDataContext<"/_root"> &
        IntersectionOfMiddlewaresResult<[]>;
    };
  }
  interface RoutesByRouteId {
    routes: "/";
  }
}
\end{minted}
\caption{Example of routes.d.ts}
\end{figure}

\textbf{Explanation Key Components}
\begin{itemize}
  \item \cc{RouteFilesByRouteId} Interface:
  \begin{itemize}
    \item Maps each \cc{route ID} (e.g., "\/\_root") to its corresponding route configuration.
    \item Includes:
    \begin{itemize}
      \item \cc{route}: Defines the route structure using the Route type, including its loader, actions, middlewares, meta function, and parent route.
      \item \cc{dataContext}: Specifies the combined data context derived from middlewares and parent routes.
    \end{itemize}
  \end{itemize}


  \item \cc{RoutesByRouteId} Interface:
  \begin{itemize}
    \item Lists all \textbf{index-only} \cc{route IDs} in the application for quick reference.
    \item Useful for navigating user to a destination within the app.

  \end{itemize}

  \item Generic Types:
  \begin{itemize}
    \item Route: The core generic type defining the structure of a route. Includes:
    \begin{itemize}
      \item Route ID.
      \item  Return type of the loader.
      \item Types for actions and middlewares.
      \item Parent route structure.
    \end{itemize}
  \end{itemize}


  \item \cc{LoaderReturnType}: Infers the return type of a loader, ensuring accurate typing for loaded data.
  \item \cc{IntersectionOfMiddlewaresResult}: Merges the output of all middlewares in the route hierarchy into a single type.
  \item \cc{GetRouteDataContext}: Retrieves the data context for a specific \cc{route ID}, combining parent and middleware data. It's added in a seperated property to avoid loop in type definition.
\end{itemize}

\subsubsection*{How This Helps in Development?}
Having a fully typed version of the router, makes it possible to write some utility types around it to make the whole app typesafe. For example, the \cc{loader} function in each route, should be typesafe and aware of it's parent data (As explained, parent loaded data should be accessible in the child loader). But how is that possible? The \cc{routes.d.ts} file is the answer. It owns the necessary types to make the loaders typesafe. For example, the \cc{LoaderArgs} is a utility type that has been written around \cc{RouteFilesByRouteId} interface (which is overridden in the \cc{routes.d.ts} file) that provides necessary types to the loader. Here is an example of how to use it:
\begin{minted}{typescript}
// pages/_root.tsx
  import { LoaderArgs } from 'ziro/router'

  export const loader = (ctx: LoaderArgs<'/_root'>) => {
    // the root's loader returns some random data like:
    return {
      version: '1.0'
    }
  }

// pages/index.tsx
  import { LoaderArgs } from 'ziro/router'

  export const loader = (ctx: LoaderArgs<'/todo'>) => {
    // ctx is fully type safe and it's parent data is accessible and typesafe
    // here ctx.dataContext.version is available and fully typesafe.
    return {
      appName: `My App v${ctx.dataContext.version}`
    }
  }
\end{minted}
\pagebreak

\subsection{Ziro Plugins}

Ziro’s plugin system is a powerful feature that allows developers to extend their applications by adding custom routes, layouts, and functionality in a modular and reusable way. By integrating directly with the \cc{manifest.json} file, plugins can introduce new pages or layouts that seamlessly integrate with the existing application’s routing structure. This system is particularly useful for building scalable applications where features can be added or modified without directly changing the core codebase.

\subsubsection{How Ziro Plugins Work}

Ziro plugins are registered in the \cc{vite.config.js} file, because they have to integrate with the framework’s manifest generation process. For example:
\begin{minted}{javascript}
import { dashboard } from './plugins/dashboard';
import { defineConfig } from 'vite'

export default defineConfig({
  plugins: [
    ziro({
      plugins: [dashboard],
    }),
  ],
});
\end{minted}

When the development server starts, Ziro detects the registered plugins and triggers their \cc{registerRoutes} method during the manifest generation process. The plugins then add custom routes or layouts to the \cc{manifest.json} file. This ensures that both server and client routers include the new additions during the file generation process.

\subsubsection{Simplified Example of a Ziro Plugin}
\begin{figure}[H]
  \begin{minted}{typescript}
import { Plugin } from 'ziro/generator';

type Config = {
  dashboardPath: string; // The route path for the dashboard page
};

export const dashboardPlugin = new Plugin<Config>(
  'dashboard', // Plugin name
  {
    registerRoutes(config) {
      return [
        {
          routeId: config.dashboardPath || '/dashboard',
          filePath: new URL(
            './dashboard.tsx',
            import.meta.url
          ).pathname,
        },
      ];
    },
  },
  {
    configPath: 'configs/dashboard.ts', // Path to the plugin's configuration file
  },
);
  \end{minted}
  \caption{Simplified Example of a Ziro Plugin}
\end{figure}

\subsubsection{Key Components of the Plugin}
The Ziro plugin system includes several core components that work together to extend the application's functionality. The plugin’s \cc{name}, such as dashboard, uniquely identifies it and ensures it can be reused and distinguished from other plugins. A configuration object (\cc{Config}) allows developers to type the plugin’s config. The \cc{registerRoutes} method defines the custom routes introduced by the plugin, including a unique identifier (\cc{routeId} or \cc{path}) and the location of the file that implements the route’s component (\cc{filePath}). Additionally, the \cc{configPath} provides a pointer to an external configuration file (\cc{configs/dashboard.ts}), which makes it possible to config the plugins based on project needs.

\subsubsection{How Plugins Modify the Manifest}
When a plugin is added, it extends the \cc{manifest.json} file with additional route definitions. For example:
\begin{minted}{json}
{
  "/dashboard": {
    "id": "/dashboard",
    "routeInfo": {
      "filepath": "./node_modules/installed-plugin/dashboard.tsx",
      "index": true,
      "hasActions": false,
      "hasComponent": true,
      "hasErrorBoundary": false,
      "hasLoader": false,
      "hasLoadingComponent": false,
      "hasMeta": false,
      "hasMiddleware": false,
      "hasLayout": false
    },
    "plugin": "dashboard"
  }
}
\end{minted}
This manifest entry ensures that the server and client routers include the new \cc{/dashboard} route when they are generated. The plugin system effectively decouples feature development from the core project, making it easier to add and manage new functionality.

Ziro plugins can be used to address a variety of practical needs, such as adding authentication systems, admin dashboards, or analytics tools. For example, a plugin can introduce login, registration, and authentication-related routes while also providing logic for securing specific pages. Similarly, admin dashboards can be added as pre-built pages and layouts, complete with tools like rate-limiting middleware for sensitive sections and their child routes. For e-commerce applications, plugins can enable features like shopping carts, product pages, and checkout workflows. These use cases demonstrate the flexibility and power of Ziro plugins to simplify development and enhance application functionality.

\subsection{Request Handling on the Server}

In Ziro, request handling on the server is a critical part of activating server-side rendering (SSR). When a request is made, the server must load the router and prepare a response based on the application’s routing logic and data-loading mechanisms. This process ensures that users receive the appropriate response, whether it’s a fully-rendered page or a partially-streamed HTML document.

\subsubsection{SSR Activation and Router Loading}
To handle requests in SSR mode, the server must render the router. This involves loading all routes by executing their associated middlewares and loaders sequentially, starting from the parent routes and proceeding to their children. The sequential execution ensures that:
\begin{itemize}
  \item \textbf{Middlewares} process the request, performing tasks such as authentication, validation, or request transformation.
  \item \textbf{Loaders} fetch the necessary data for each route, building the context needed for rendering the page.
\end{itemize}
If any middleware or loader throws a response (e.g., a redirect or error response), the server immediately sends this response back to the user without further processing. This mechanism guarantees precise handling of redirects, errors, and other special cases.

\subsubsection{Handling SSR and Partially-SSR Modes}
\begin{itemize}
  \item \textbf{SSR Mode}: In SSR mode, the server waits for all routes to finish their loading processes. This ensures that the full HTML document is ready before sending it to the user. While this approach ensures a complete response, it may increase the initial response time.
  \item \textbf{Partially-SSR Mode}: In partially-SSR mode, the server begins streaming the HTML to the user as soon as possible. Instead of waiting for all routes to complete their loaders, it sends the initial HTML structure and streams additional content as it becomes available. This approach significantly reduces the perceived loading time for users, improving their experience.
\end{itemize}
In both modes, if no error or custom response is thrown during the loading process, the server proceeds to render the router using \cc{renderToReadableStream} (from React library). This method generates a stream of HTML that is sent to the user incrementally.

\subsubsection{Integration with the Web Server}
To manage these processes, Ziro integrates with a web server. The chosen server framework, \textbf{H3}, handles the HTTP requests and responses while coordinating with Ziro’s router logic. This integration ensures smooth and efficient delivery of server-rendered pages and guarantees that the server works regardless of the Vite's development server as in the production environment, the server is responsible for handling all requests and responses without Vite.

\subsubsection{Why H3 as the Web Server?}

H3 was chosen as the web server for Ziro due to its modern and lightweight design, making it an ideal fit for handling the needs of a high-performance SSR framework. H3 is a runtime-agnostic web framework built on top of popular technologies that offer both flexibility and performance.

A key feature of H3 is its runtime-agnostic nature, which provides support for different types of request handlers. In Ziro, this means that H3 can operate seamlessly on top of the standard Web \cc{Request} and \cc{Response} interfaces, ensuring full compatibility and a consistent handling experience. This allows Ziro to process HTTP requests and responses efficiently while supporting SSR logic, middleware execution, and data loading in a streamlined manner.

Another significant advantage of using H3 is its capability to deploy applications across different runtimes without the need for multiple adapters. Whether deploying on \cc{Cloudflare workers/pages}, \cc{Bun}, or \cc{Deno}, H3 handles the compatibility, simplifying the deployment process and making it easier to maintain cross-runtime consistency.

Overall, H3's efficient handling of HTTP requests, runtime flexibility, and compatibility with modern web standards make it the perfect choice for supporting Ziro’s request handling and server-side rendering capabilities.

\subsection{React integrations}

Ziro’s core router library, \cc{ziro/router}, is designed as a headless routing solution, which means it handles routing logic without any built-in user interface (UI). To bring this routing logic to life in a web application, a UI integration is necessary. This is where \cc{ziro/react} comes into play, making it possible to connect Ziro’s routing capabilities with React applications.

\subsubsection{Setting Up React Integration}
To integrate Ziro with React, the client-side entry point of the application is typically a JSX file that renders the router to the DOM. Here is an example of how the main entry file might look:
\begin{minted}{javascript}
import { hydrateRoot } from 'react-dom/client';
import { Router } from 'ziro/react';
import router from '/.ziro/router.client.ts';

hydrateRoot(document, <Router router={router} />);
\end{minted}

In this setup, the \cc{hydrateRoot} method from React is used to render the router into the existing HTML structure. The Router component from \cc{ziro/react} takes the router prop, which is imported from the generated \cc{router.client.ts} file. Once the router is loaded and ready, it will handle the rendering of the route components on the page, making the application fully interactive.

\subsubsection{Key Utilities for React Integration}
To make Ziro compatible with the React ecosystem, \cc{ziro/react} provides several utilities that enhance its integration:
\begin{itemize}
  \item \textbf{\cc{Link} Component}: This component allows for seamless navigation between pages within the application. It functions similarly to the Link component from \textbf{React Router}\footnote{\url{https://reactrouter.com/start/library/navigating\#link}}, making it intuitive for developers familiar with React.
  \item \textbf{\cc{useAction} Method}: The useAction method provides a way to call route actions directly from React components. This utility helps handle user interactions that need to trigger route-based actions, such as form submissions or other state-changing events fully type-safe.
  \item \cc{Outlet} Component: To render a child route within its parent route, Ziro uses the \cc{<Outlet />} component, similar to React Router's implementation. When a request is made, the framework identifies the route tree, which includes all parent and child routes from the top-level (\_root.tsx or \_layout.tsx) to the specific leaf route. The \cc{<Outlet />} component determines where the child routes should be rendered within the parent route’s layout. This ensures a seamless rendering hierarchy, allowing developers to define layouts and nest routes effectively.
\end{itemize}

\subsection{Client-Only Server-Only Transformation}

When using SSR (Server-Side Rendering) in Ziro, it's essential to ensure that server-only code, such as database connections and private API calls, is not exposed to the client-side JavaScript. Ziro achieves this by integrating Vite and \textbf{Babel}\footnote{\url{https://babeljs.io/}} to transform route modules and control what is sent to the client.

Vite, which is based on ECMAScript modules (ESM), provides a fast and efficient development server by transforming modules on demand. This means that Vite only processes code when a module is requested by the browser, leading to quicker development and optimized builds. Ziro's Vite plugin takes advantage of this by adding a custom step to the transformation process. When a route module is requested, the plugin uses Babel to scan and remove any server-specific code, such as loaders, actions, and middlewares, ensuring that only client-compatible code remains.

This approach protects sensitive server-side operations, such as database queries or internal API interactions, from being exposed in the client-side bundle. It maintains the security of the application while delivering an optimized, lightweight JavaScript bundle for the browser. By leveraging Vite's ESM-based transformation and adding custom filtering logic with the Ziro plugin, this method enhances both performance and security.


\subsubsection{Caching, the Shared Layer Between Server and Client}

Data caching plays a crucial role in synchronizing server-side rendered (SSR) routes with the client to ensure efficient data transfer and rendering. In Ziro, this mechanism is designed to optimize data flow between the server and client, providing a seamless and fast user experience.

In SSR mode, the server is responsible for running the middlewares, loaders and actions to load data. When a client makes a request for a route, the client router sends a request to the server to retreive the necessary data. The server processes this request, executes the necessary middlewares and loaders and loads the relevant data using the route loaders. During this data loading phase, the server generates a cache object containing the route data and responses. This cache object is then serialized as JSON and sent back to the client router. The client router receives this cached data and directly loads it into its own router cache. The client router can then use the data stored in its cache to render the components.

In contrast, in Client-Side Rendering (CSR) mode, data handling is managed entirely on the client. Since the server only serves a minimal HTML shell and static assets, all loaders, middlewares, and data-fetching operations are executed within the client's browser. The client fetches data as needed using JavaScript running in the browser, so there is no need for a data synchronization process from the server. This approach simplifies the interaction and reduces server load but may result in longer initial load times compared to SSR.

This data caching strategy in SSR mode ensures consistency between the server-rendered output and the client's rendered state, providing a smooth transition when the page is loaded. By reusing the data already fetched and processed by the server, the client can render routes quickly without additional network requests.


\subsection{How to setup a project using Ziro}

Setting up a project with Ziro involves creating a React-based Vite project, installing the Ziro package, and configuring both Vite and TypeScript to integrate Ziro's routing and rendering system. Below, we'll walk you through the necessary steps to get your Ziro project up and running.

\begin{enumerate}
  \item \textbf{Create a React-Vite Project} First, create a new Vite project with the React template using the following command:
  \begin{minted}{bash}
    npm create vite@latest my-ziro-app -- --template react-ts
  \end{minted}
  This command will create a new React project with TypeScript support.

  \item \textbf{Install the Ziro Package} Navigate to your project directory and install the Ziro package:
  \begin{minted}{bash}
    npm i ziro
  \end{minted}
  \item \textbf{Configure vite.config.ts} Set up your \cc{vite.config.ts} file to include Ziro as a plugin:
  \begin{minted}{typescript}
import react from '@vitejs/plugin-react';
import { defineConfig } from 'vite';
import ziro from 'ziro/vite';

export default defineConfig({
  plugins: [
    react(),
    ziro(),
  ],
});
\end{minted}
This configuration ensures that Vite runs with Ziro’s plugin.

\item \textbf{Update tsconfig.json} To make sure TypeScript recognizes Ziro's type declarations, modify your \cc{tsconfig.json} to include the Ziro type definitions:
\begin{minted}{json}
  {
  "compilerOptions": {
    "strict": true,
    "esModuleInterop": true,
    "jsx": "react-jsx",
    "module": "esnext",
    "types": [
      "vite/client",
      "./.ziro/routes.d.ts"
    ],
    "moduleResolution": "Bundler"
  }
}
\end{minted}

\item \textbf{Create Route Files}: Create the pages directory and add the following route files.
  \begin{minted}{typescript}
// _root.tsx
import { FC, PropsWithChildren } from 'react';
import { Body, Head, Html, Outlet, RouteProps } from 'ziro/react';
import baseStyle from './styles.css?url';

export default function Root(props: RouteProps<'/_root'>) {
  return <Outlet />;
}

export const Layout: FC<PropsWithChildren> = ({ children }) => {
  return (
    <Html>
      <Head>
        <meta charSet="utf-8" />
        <meta name="viewport" content="width=device-width, initial-scale=1" />
        <link href={baseStyle} rel="stylesheet" />
      </Head>
      <Body>{children}</Body>
    </Html>
  );
};
  \end{minted}

  \begin{minted}{typescript}
    // index.tsx
import { MetaFn } from 'ziro/router';

export const loader = async () => {
  return {};
};

export const Loading = () => {
  return <span>Loading home page...</span>;
};

export const meta: MetaFn<'/'> = async () => {
  return {
    title: 'Homepage',
  };
};

export default function Index() {
  return (
    <div>
      <h1>Hello World!</h1>
    </div>
  );
}
      \end{minted}
      These files define the main route layout and the homepage, including components and metadata.


      \item \textbf{Update package.json} Scripts Modify the scripts section in your package.json to run the project with Ziro:
\begin{minted}{json}
"scripts": {
  "dev": "ziro dev",
  "build": "ziro build",
  "preview": "vite preview"
}
\end{minted}
This configuration ensures that running \cc{npm run dev} starts the development server with Ziro, \cc{npm run build} generates a production build, and \cc{npm run preview} previews the built application.


\end{enumerate}

With these steps, you now have a React-Vite project set up to work with Ziro, enabling server-side rendering, routing, and type safety out of the box. This setup provides an efficient development experience with modern tools and seamless server-client interactions.

\pagebreak
