\section{Introduction}

Web development frameworks have evolved significantly in recent years, yet many still lack a \hl{cohesive} approach to combining scalability, modularity, and developer-friendly abstractions. Frameworks like Next.js and Remix provide powerful tools for building web applications, but they do not expose certain critical layers, such as the router, for plugin-based customization. This limitation often forces developers to repeatedly create essential features like authentication, which involves setting up routes, middleware, actions, and components manually for every project.

Ziro addresses this gap by introducing a thoughtful approach that abstracts multiple parts of a full-stack framework while allowing them to scale independently. Ziro enables plugins to dynamically add custom pages, endpoints, middleware, and actions. This approach significantly reduces repetitive work and accelerates the development process, making it ideal for tasks like implementing reusable authentication systems or integrating predefined pages across projects.

Additionally, Ziro \hl{adheres} to web standards to ensure compatibility across various runtimes, including Node.js and browsers. The framework is structured into separate packages to maintain modularity and framework-agnostic design:

\begin{itemize}
	\item ziro/router: Contains the core routing concepts, designed to work independently of any specific framework.

	\item ziro/react: A React-compatible implementation for seamless integration with React-based projects.

	\item ziro/generator: A tool for generating TypeScript type-safe definitions to ensure robust developer workflows.
\end{itemize}

Ziro’s features are \hl{grounded} in modern web development principles. It offers a fully typesafe routing system, ensuring that all actions and data derived from middleware or layouts maintain type safety throughout the application. With its sequential data flow, loaders from nested pages run in order, allowing data fetched at higher levels to cascade to lower ones. The plugin system further enhances flexibility by simplifying the process of extending applications with custom functionalities.

Targeted primarily at developers, Ziro also opens the door to building tools for broader audiences, such as WordPress users who may not have a strong technical background. By introducing modularity and preconfigured components, it streamlines the creation of dynamic, scalable applications like e-commerce platforms, personal blogs, and full-stack applications.

Developing Ziro has presented several challenges, including implementing robust type safety—a feature still relatively novel in modern frameworks—and ensuring seamless integration across different rendering modes: SSR, CSR, and a partially SSR mode that streams data to users. The result is a proof of concept that not only demonstrates the feasibility of combining full-stack JavaScript apps with plugin-based scaling but also provides a modular foundation for scalable, dynamic application development.

This introduction outlines Ziro’s core philosophy, features, and potential impact, setting the stage for a deeper exploration of its architecture, methodology, and applications.

% \begin{minted}{javascript}
% function greet(name) {
% 	const message = `Hello, ${name}!`;
% 	console.log(message);
% }
% greet("Nariman");
% \end{minted}


\pagebreak
