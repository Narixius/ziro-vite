\section{Future Works} % 3pages

The future works listed bellow can help Ziro evolve into a better and powerful framework that meets the needs of modern web development, making it more robust, flexible, and suitable for a wider range of applications. Here are some potential directions and enhancements that can be included to expand the capabilities of Ziro and make it more robust and useful:

\begin{itemize}
	\item  \textbf{Improved Plugin System}: Enhance the plugin system to support more complex functionalities, such as dynamic route generation, general middlewares, general actions and having more control over the requests for the middlewares, loaders and actions. This would make it easier for developers to create and integrate more sophisticated plugins.

	\item \textbf{CLI Tool for Project Initialization}: Develop a command-line interface (CLI) tool similar to Vite's, allowing developers to create pre-configured Ziro projects easily. This tool could be invoked with commands like \cc{npm create ziro}, streamlining the setup process and ensuring best practices are followed from the start.

	\item \textbf{HMR Support for Ziro in Vite}: Adding first-class Hot Module Replacement (HMR) integration to Ziro's Vite plugin will enable real-time updates to routes and components without a full page reload, improving development speed and maintaining app state.

	\item \textbf{Production Build Support}: Develop and integrate the production build functionality to ensure that Ziro-based projects are production-ready.

	\item  \textbf{Cross-Platform Compatibility}: Expand Ziro's compatibility to support at least one additional runtimes, such as serverless environments and various cloud hosting providers (e.g., \cc{Cloudflare workers/pages}, \cc{Netlify}). This would enable developers to deploy Ziro-based applications more flexibly.

	\item \textbf{React 19 Support}: Ensure compatibility with the latest version of React (React 19). This includes updating dependencies, testing for breaking changes, and leveraging new features and improvements introduced in React 19 to enhance the overall developer experience.

	\item  \textbf{Server-Side Caching}: Implement server-side caching mechanisms to improve performance, reduce redundant data fetching, and enhance the overall response time of Ziro-powered applications.

	\item \textbf{Testing \cc{ziro/react} Integration}: Since \cc{ziro/react} is crucial for connecting the Ziro router with React applications, it needs to be well-tested to make sure new changes doesn't break anything.

	\item  \textbf{Real-World Solutions}: Implementing solutions and examples for common needs like authentication, dashboards, and i18n will make Ziro more practical. These examples will help developers quickly integrate essential features into their projects.

	\item  \textbf{Documentation and Community Support}: Provide more comprehensive guides, tutorials, and examples for users (developers).

	\item  \textbf{Performance Benchmarking}: To ensure that Ziro performs well under various loads and use cases. This would help identify potential bottlenecks and refine the performance of the framework.

\end{itemize}

\pagebreak
