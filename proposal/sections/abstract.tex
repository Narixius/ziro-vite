\section{Abstract}

Web development has evolved to require frameworks that are both flexible and efficient. Existing server-side rendering (SSR) frameworks, such as Next.js and Nuxt.js, offer powerful solutions but can be less suited for building modular systems with pre-integrated features like authentication and administrative tools. This project presents Ziro, an SSR framework designed to explore the feasibility of creating a plugin-driven architecture within the JavaScript ecosystem.

Ziro provides a practical foundation for building dynamic web applications by emphasizing modularity and extensibility. Its architecture enables developers to integrate pre-built plugins, reducing repetitive tasks and focusing on core functionality. At its core, the \texttt{ziro\/router} provides a cross-environment compatible router, the \texttt{ziro\/generator} automates TypeScript type generation to offer robust type safety and \texttt{ziro\/react} brings React.js compability to the router. The framework also supports hybrid rendering modes, combining server-side and client-side rendering to enhance performance and user experience.

Developed iteratively with rigorous testing using Vitest, Ziro incorporates features refined through real-world scenarios and edge case handling. By blending flexibility with a plugin-driven architecture, Ziro addresses common development pain points and establishes a foundation for future advancements in modular web frameworks. This proposal details the development process, key technical components, and the insights gained from building Ziro.



\pagebreak
