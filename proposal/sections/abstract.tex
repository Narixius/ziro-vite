\section{Abstract}

Web development has evolved to require frameworks that are both flexible and efficient. Existing server-side rendering (SSR) frameworks, such as Next.js and Nuxt.js, offer powerful solutions but can be less suited for building modular systems with pre-integrated features like authentication and administrative tools. This project presents Ziro, an SSR framework designed to explore the feasibility of creating a plugin-driven architecture within the JavaScript ecosystem.

Ziro is built to provide a practical foundation for developing dynamic web applications. Its modular structure allows for the addition of pre-built plugins, aiming to reduce repetitive work for developers. Key components like the \texttt{ziro/router} ensure compatibility across environments, while the \texttt{ziro/generator} supports type safety through automated TypeScript type generation. Ziro also integrates server-side and client-side rendering to improve both performance and user experience.

The framework has been developed iteratively, with a focus on testing edge cases using Vitest and refining features based on practical needs. By combining these elements, Ziro seeks to address common development challenges while providing a basis for future improvements. This proposal outlines its development process and the lessons learned from building a modular web framework.
\\
\\
\\
\\


\pagebreak
